\documentclass{acm_proc_article-sp}

\usepackage{lipsum}

\title{Predicting Stock Trends using Twitter Data}
\author{
    Nicolas Broeking \\
    \small Department of Computer Science \\
    \small University of Colorado Boulder \\
    \small \textbf{CSCI 5502} \\
    \small nicolas.broeking@colorado.edu \\
    \and
    Anna Hoffee \\
    \small Department of Applied Math \\
    \small University of Colorado Boulder \\
    \small \textbf{CSCI 4502} \\
    \small anna.hoffee@colorado.edu \\
    \and
    Joshua Rahm \\
    \small Department of Computer Science \\
    \small University of Colorado Boulder \\
    \small \textbf{CSCI 5502} \\
    \small joshua.rahm@colorado.edu \\
}


\begin{document}

\maketitle

\begin{abstract}
With social media on the rise, the amount of data available for processing is
growing at an increasing rate. It is an advanced area of research to be able to
use that data to predict or better understand the world around us. In our
project we will attempt to use Twitter to determine if, and how strong of, a
relationship exists between public sentiment and chosen stock values.

\end{abstract}

\section*{Categories and Subject\\ Descriptors}

H.3.4 [Information Systems Applications-Systems and Software]: Information
networks; J.4 [Social and Behavioral Sciences]: Economics

\section*{General Terms}
Algorithms, Measurement, Economics, Experimentation, Human Factors

\section*{Keywords}

Social Networks, Trends, Blogging, Tweets, Hashtag, Twitter, Stock

\section*{Goal}

To determine if there is a relationship between the sentiment of tweets and the
change in stock price for the next day. Then to create a user friendly mobile
application to display real time stock predictions.

\section{Introduction}

The recent rise in popularity of online social media has led to a huge amount
of social data available online for analysis. The popular social media site,
Twitter, gets an average of 6,000 tweets posted per second and 500 million per
day. These tweets can be analyzed to determine future market values. For the
purposes of this project we looked at four companies, Amazon, Apple, Google,
and Samsung.o  

The first step is to open a connection to the twitter streaming api. Using this
api we receive anywhere between 1\% and 40\% of all the tweets being tweeted.
This percentage depends on the current Twitter load. We collect the stock
information from yahoo finance. We store the stock data for the same time
period we get the stock data for multiple granularities.

The next phase in our pipeline is the data storage step. We are using sqlite3 as
our database to store all of our tweets. Each tweet stores information
related to the user and the tweet to determine possible variables to match
against for the statistical analysis.

The third phase is to pre-process the data. Before we process the data we need
to filter based on language. We only accept tweets that are English and have no
images or videos. Once we have our English tweets we start categorizing the
tweets. We separate each tweet into four different bins, one for each company.
We then change all URLs to URL and all hash tags to TAG. This allows us to use
these as features for the sentiment analysis. 

The fourth phase is our sentiment analysis phase. We use a naive Bayesian
classifier to detect whether a tweet is positive. Once we have applyed our
sentiment analysis to the data send it to our next stage.

We are currently working to implement this phase. We are taking the data and
trying to detemine a relationship between the sentiment and a change in the
stock price.

Our next step is to determine the realtionship between the variables and then
implemnt a mobile application that recomends which company to invest in. 

\section{Authors}

The authors of this proposal are Nicolas Broeking, Anna
Hoffee, and Joshua Rahm. Broeking and Rahm are graduate students at the
University of Colorado at Boulder. Hoffee is a undergraduate student at the
University of Colorado. Nicolas Broeking has worked on embedded systems and
mobile applications for the past four years. He would like to take the results
from the project and create a user friendly application that allows a user to
easily interact with the data and make financial predictions.  Josh Rahm has
spent his career working with embedded devices, cellular technologies and
billing platforms, and is interested in applying data mining concepts to
markets.  

\section{Motivation}

The Stock Market is one of the largest entities in Western and World economies.
In a world where a 15\% increase is assets per year is massive,  even the
ability to increase certainty in the market by a few percent is a huge.
Companies and individuals can harness this technology make billions and secure
investments, producing and saving billions for the economy. While making
billions is outside the scope of this project, we think it is possible to
significantly increase the accuracy of stock predictions. We know that public
image is critical to stock value. If a company's image drastically decreases
then we predict that its stock value will change. It is our goal to discover
how the company's public image, determined through Twitter, will affect its
stock price and then to create an application that recomends stocks for a user.

\section{Literature Survey}

Many similar experiments have been done in the
field of data mining. The first and probably the closest to our project is
Correlating Financial Time Series with Micro-Blogging. This project looks at
150 random companies stock prices over a six month time period. Then they take
all the tweets with specific hashtags related to the company and construct a
context graph. In this graph the tweets themselves were nodes and any actions
on the tweets were edges. They then use this graph to find relationships with
the stock price. This project looked for other things than just stock trends
though. They wanted to find relationships between the data and how much a stock
will change, and what the values of the stock will be. They determined that the
most reliable way to determine the information is by looking at the number of
edges in the graph.[1] 

Another team, Twitter Mood Predicts the Stock Market, attempted to determine
how twitter “mood” effects the stock price. They grouped tweets into 6
dimensions Calm, Alert, Sure, Vital, Kind, and Happy. They then took these
categories and created a relationship between to the closing value of the Dow
Jones. They found that there is a correlation between what people are posting
and how the stock market as a whole performs. [2] 

Many other studies have been done similar to the first two on how to use
twitter to predict stock prices. The one thing that has been determined for
sure is that there is a high correlation between peoples' attitudes and stock
value. Twitter was used in 2013 to predict a drop in Royal Caribbean's stock
value when people started tweeting about the flu spreading on one of its
cruises. Researchers were able to predict the drop in price 48 minutes before
the stock plunged about 3\%.[3]

\section{Tasks}
In order to achive this task we split our project up into 7 tasks.

\subsection{Data Collection} 
\subsubsection{Stock Collection} 

Yahoo Finances allows us to gather historical stock data for each of the four
companies.  This data source provides us with the Open, Close, High, Low, and
Data attributes. We use this information to find a correlation between the
sentiment of the tweets and the change in stock price.

\subsubsection{Twitter Collection} 

Twitter data is collected from the Twitter Streaming api. This api provides us
anywhere from 4\% to 40\% of all tweets being posted depending on the current
load that twitter has to handle. As apart of the data collection phase we
filter our tweets. If a tweet contains any reference to any of the compainies
we sort it. This allows us to dramatically decrease the amount of tweets that
we are going to store. After all of our filtering we still have a massive
amount of data. Each week ends up containing about 2GB of data. This then
brings the total amount of tweets to over 10GB for about 5 weeks of collecting
tweets. 

We are storing these tweets in a sqlite3 database. We decided to go with a
sqilte3 database because it is fast and portable. We choose a relational
database schema because we needed to optimize our next phases for speed. Beause
we could have over 20 gb by the end of the semester we need a way to be able to
easily and quickly read data from the database. For each tweet, we store:  if
the tweet was re-tweeted, how many times the tweet was re-tweeted, the date and
time, the user id, the tweet id, the users followers count, the users friend
count, the user's name and finally the tweets text. 

\subsection{Data Preprocessing}

In the preprocessing step we need to filter our tweets on certain criteria. The
first thing we do is filter all tweets based off of if they contain a reference
to one of the four companies. This step is logically in the Data Preprocessng
step however we perform it during the collection to try and reduce the volume
of tweets.  The next step is to filter based off of language. Because analyzing
tweets in multiple languages makes the problem very difficult we only store
english tweets. The final step in pre processing the data is to change all URLs
in the tweet to URL and to change all hashtags to TAG. This allows us to create
a feature that we can use to do sentiment analysis in the next step.

\subsection{Sentiment Analysis}

Once we have preprocessed all of the tweets we can finally analyze the text
segment. To perform sentiment analysis we use a naive Bayes classifier. We
train the classifier on 5000 tweets and then test its accuracy with 10000
tweets.  We are only interested in the ratio between positive tweets to total
tweets so we trained our classifier to mark anything that was not obviously
positive as negative. We did not need to include a neutral category. 

Performing sentiment analysis on tweets proved to be a much harder task than
initially thought. We experimented with using different features such as
letters, user attributes, and tweet attributes but for the most part these led
to very low accuracies. Finally we found that taking the 50 most positive words
from that training set and using these as features for the classifier yielded
the highest accuracy. 

The accuracy of our classifier is about 64\%. We were
not initially satisfied with this accuracy however in our research to create a
better twitter sentiment engine we talked with Jamey Wood, CTO of Wayin. Wayin
specializes in twitter sentiment analysis. 64\% is acctually a very high
accuracy for this kind of analysis in industry and so increasing the accuracy
much higher seems to be an impossible task.
 
\subsection{Correlation Analysis}

Now that the sentiment analysis is completed and we have all the tweets classified as positive or negative we can begin the correleation analysis. We combined the twitter data with historical stock data. The stock data is obtained from http://www.nasdaq.com/quotes/. Specefically we are looking at Open and Close values. Before starting correlation analysis the twitter data and stock had to be combined. This was done all in R. The next step is to regress the change in open and close values for a given day on the percent of positive tweets about the company from the previous day. To be able to do this the twitter and stock data need to be matched up correctly. Because there are days of missing twitter data that cannot be recovered and no stock data on weekends there was some pre-processing involved in matching up tweets from one day with open and close values for the next day. Now we have an R script that is generalized to any set of stock and twitter data with arbitrary patterns of missing days, so processing of further data will be quicker. Currently there aren't enough data points to come to a good conclusion about, but soon there will be a lot more.

Looking ahead, after more data is collected the regression equation will be as follows: $ \ \ Y = \beta_0 + \beta_1X $ where $Y$ is the change in open and close values, and $X$ is the percent of positive tweets from the previous day. We are anticipating to see positive correlation between the percent of positive tweets and the change in open and close values. To test this correlation we'll use the chi-squared test for independence, the lift calculation, and the t-test to test the hypothesis that $\beta_1 = 0$ indicating that $X$ is independent of $Y$. To fit the regression model we will use the ordinary least squares estimator. We're performing this analyis on data from Google, Samsung, Apple, and Amazon. As well I anticipate that the stock data/twitter data for each of those companies will not be independent of each other, so they could potentially be added as regressors in each others models. Once more data is collected we can begin the statistical analysis. 

\subsection{Error Analysis}

To evaluate the accuracy of the model we'll look at the $R^2$ value, which shows the percent change in $Y$ that is explained by $X$. We can also look at the standard errors the of regression coefficients and the residuals of the models. If there is lots of correlation in the errors we can try using the generalized least squares estimators instead. 


\subsection{Application}

Our goal is to create a mobile application that can process tweets in real time
and then display what companies it recomends investing in. In order to
acomplish this task. We will take our model that we developed using the stages
above and using a server to analyze tweets in real time. The server will then
use the model to predict the stock values for the next day. When the user opens
the app on his phone the app will reach out to the server using a web request
and get the recomended stocks using json. The mobile device will then display
these results to the user.

\subsection{Sources}

JOSH CAN YOU USE BIBTEX

 \end{document}
